\section{Assumptions}
In order to provide context around the system we make assumptions regarding it. We make assumptions regarding the company size and it's production need. Additionally, we also make some assumptions about the production process in regards to technology and requirements.

We assume the the company is a mid-sized company producing bikes and electric bikes for the commercial market. We have made the assumption that it is a mid sized company because the hardware seems expensive and because most smaller companies have no need for a 24/7 production line. Moreover, it does not appear to be a large company as the production line appears small. The assumption that the company produces bikes or electric bikes was made because the tools in the assembly line seems advanced, however this assumption is mostly arbitrary but serves to give an interesting context for the system.

Based on the assumption that the company produces bikes and electric bikes, we further assume that the production line should be able to manipulate parts from metal and plastics to electronics, in a complex assembly and treatment process. In the process of this we assume that there is a need for many different technologies along the assembly line, which we assume would create a strong need for middleware.
We also assume that the production line should concern itself with being able to recover unfinished products that may come from an emergency stop or an unexpected breakdown. In such cases, we assume that it would be reasonable for it to automatically recover or discard the products when resuming operations.
Furthermore, it would be important to allow for the addition or variation of modules in the assembly line. This would be important in order to accommodate changing production needs.
We assume that  system needs to concern itself with security and safety of human actors during production due to compliance. This could cause a need for emergency stops or automatic proximity shutdowns.
